\begin{abstract}

\end{abstract}

%\begin{keyword}
\keywords{}
%\end{keyword}

\MSC{Primary: 62G20, 62E20 Secondary: 62-08}

\section{Introduction}\label{Sec:Intro}
	The goal of this manuscript is showing the asymptotic behavior of the kernel distance when the parameter of the family is estimated from the data (data-driven parameter). Recall first some notation: $\left\{k_{\lambda}:\lambda\in\Lambda\right\}$ is a family of kernels where $\Lambda$ is a parameter space to be specified later. For each of the kernels $k_{\lambda}$ we will denote by $\mathcal{H}_{k,\lambda}$ its associated RKHS and the unit ball of such space as $\mathcal{F}_{k,\lambda}$. For a given Borel's measure $\operatorname{S}$, the mean embedding is defined as
	\begin{equation}
		\mu_{\operatorname{S}}(\cdot)=\int_{\mathcal{X}}k_{\lambda}(\cdot,y)\,\operatorname{d}\!\operatorname{S}(y),
	\end{equation}
	where the integral is understand in the Pettis' sense. The interest of mean embedding lays in the following definition property: for every $f\in\mathcal{H}_{k,\lambda}$, we have that $\operatorname{S}(f)=\inp{f}{\mu_{\operatorname{S}}}{\mathcal{H}_{k,\lambda}}$, where $\inp{\,}{\,}{\mathcal{H}_{k,\lambda}}$ denotes the inner product. In terms of the Riesz's representation theorem for Hilbert's spaces, the mean embedding is the dual element of the integral functional induced by $\operatorname{S}$ in $\mathcal{H}_{k,\lambda}$ (provided integrability assumptions).
	
	I propose the following variation of the set of assumptions used in \textcite{Carcamo2024}.
	\begin{description}
		\labitem{(Reg)}{itm:Reg} \textit{Regularity assumption.} $\mathcal{X}$ is a separable metric space and each kernel is continuous as a real function of one variable (with the other kept fixed).
		%, that is, $\sup_{x\in\mathcal{X}} k(x,x)$, $\sup_{x\in\mathcal{X}} k_\lambda(x,x)<\infty$ and $k_\lambda(x,\cdot)$, $k(x,\cdot)$ are continuous functions, for each fixed $x\in \mathcal{X}$, $\lambda\in\Lambda$.
		\labitem{(Dnk)}{itm:Dnk} \textit{Dominance assumption.} There exists a constant $c>0$ such that $k_{\lambda}\ll c\,k$, for all $\lambda\in\Lambda$. Further, $k$ is $\operatorname{L}^{2}(\operatorname{P}+\operatorname{Q})$ on the diagonal, that is, $\int_{\mathcal{X}}k(x,x)\,\operatorname{d}(\operatorname{P}+\operatorname{Q})(x)<\infty$.
		\labitem{(Ide)}{itm:Ide} \textit{Identifiability assumption.} If $\operatorname{P}\neq\operatorname{Q}$, there exists $\lambda\in\Lambda$ such that $\mu_{\operatorname{P}}^{\lambda}\neq\mu_{\operatorname{Q}}^{\lambda}$.
		% The mean embeddings separate $\P$ and $\Q$ whenever , that is,
		\labitem{(Par)}{itm:Par} \textit{Continuous parametrization.} {\color{orange}$\Lambda$ is a compact subset of $\mathbb{R}^{k}$ (with $k\in\mathbb{N}$)}, for a fixed $(x,y)\in\mathcal{X}\times\mathcal{X}$, the function $\lambda\mapsto k_{\lambda}(x,y)$ is differentiable from $\Lambda$ to $\mathbb{R}$ with derivative $\partial k_{\lambda}(x,y)$ and there exists positive functions $G_{1}\in\operatorname{L}^{1}(\operatorname{P}+\operatorname{Q})$ and $G_{2}\in\operatorname{L}^{1}\left((\operatorname{P}+\operatorname{Q})^{2\,\otimes}\right)$ such that $\underset{\lambda\in\Lambda}{\operatorname{sup}}\,\left(k_{\lambda}(x,x)\right)\leq G_{1}(x)$ for $\operatorname{P}+\operatorname{Q}$-a.s $x\in\mathcal{X}$ and $\underset{\lambda\in\Lambda}{\operatorname{sup}}\,\left(\left|\partial k_{\lambda}(x,y)\right|\right)\leq G_{2}(x,y)$ for $(\operatorname{P}+\operatorname{Q})^{2\,\otimes}$-a.s $(x,y)\in\mathcal{X}^{2}$.
		\labitem{(Sam)}{itm:Sam} \textit{Sampling scheme.} The sampling scheme is balanced, that is, $\frac{n}{(n+m)}\to\theta$, with $\theta\in[0,1]$, as $n,m\to\infty$.
	\end{description}
	{\color{orange}Desaparece (Dom) y aparece (Dnk) para mostrar las propiedades suficientes para el la inclusi\'{o}n en un RKHS m\'{a}s grande de funciones continuas y aplicar \textcite{Marcus1985}.
	
	Modificamos (Par) para que incluya la diferenciabilidad.}

	We will exploit the following Prof. C\'{a}rcamo's idea: if
	\begin{equation}\label{Eqn:ProfCarcaIdea}
		\begin{aligned}
			\psi(\lambda,\operatorname{P}-\operatorname{Q})&=\underset{f\in\mathcal{F}_{k,\lambda}}{\operatorname{sup}}\,((\operatorname{P}-\operatorname{Q})\,(f))=\left\|\mu_{\operatorname{P}}^{\lambda}-\mu_{\operatorname{Q}}^{\lambda}\right\|_{\mathcal{H}_{k,\lambda}}
			\\&=\left(\int_{\mathcal{X}}\int_{\mathcal{X}}k_{\lambda}(x,y)\,\operatorname{d}(\operatorname{P}-\operatorname{Q})(y)\,\operatorname{d}(\operatorname{P}-\operatorname{Q})(x)\right)^{\rfrac{1}{2}},
		\end{aligned}
	\end{equation}
	we can use the integral expression to compute the derivative explicitly.
	\begin{equation}\label{Eqn:Witness}
		h^{+,\lambda}=\frac{\mu_{\operatorname{P}}^{\lambda}-\mu_{\operatorname{Q}}^{\lambda}}{\left\|\mu_{\operatorname{P}}^{\lambda}-\mu_{\operatorname{Q}}^{\lambda}\right\|_{\mathcal{H}_{k,\lambda}}},
	\end{equation}
	
	
	
	\begin{equation}\label{Eqn:Lambda_union_unit-ball}
		\begin{aligned}
			\mathcal{F}_{k,\lambda}&=\left\{f\in\mathcal{H}_{k,\lambda},\ \|f\|_{\mathcal{H}_{k,\lambda}}\leq1\right\},
			\\
			\mathcal{F}_{k,\Lambda}&=\bigcup_{\lambda\in \Lambda}  \mathcal{F}_{k,\lambda}.
		\end{aligned}
	\end{equation}
	\begin{equation}
		\rho\left(f_{1},f_{2}\right)=\underset{\operatorname{S}\in\{\operatorname{P},\operatorname{Q}\}}{\operatorname{max}}\,\left(\left(\int_{\mathcal{X}}\left|f_{1}(x)-f_{2}(x)\right|^{2}\,\operatorname{d}\!\operatorname{S}(x)\right)^{\rfrac{1}{2}}\right)
	\end{equation}
	\subsection{Preliminaries}
		\begin{itemize}
			\item Reproducing Kernel Hilbert Spaces.
			\item Mean embedding.
			\item Empirical process.
			\item U-statistics.
			\item Discussion on the joint process $\left(a_{m,n}\,\left(\lambda_{m,n}-\lambda\right),\mathbb{G}_{m,n}\right)$ on $\Lambda\times\mathcal{F}_{k,\Lambda}$. More precisely: sufficient conditions and M-estimators.
		\end{itemize}
		\subsubsection*{Generalized mean embedding}
			{\color{orange}Las aplicaciones continuas y prelineales satisfacen la mayor\'{i}a de propiedades que les pedimos a las medidas de Borel $\mathcal{MB}_{\operatorname{p}}(\mathcal{X})$ para que haya mean embedding.}
			\begin{Lema}
				Let us assume that the family of kernels $\left\{k_{\theta}:\theta\in\Lambda\right\}$ satisfies \ref{itm:Par}. Then, for every $\lambda\in\Lambda$, $\mathcal{C}_{\operatorname{bpl}}\left(\mathcal{F}_{k,\Lambda},\rho\right)\subseteq\mathcal{H}_{k,\lambda}^{\ast}$ and  $g\in\mathcal{C}_{\operatorname{bpl}}\left(\mathcal{F}_{k,\Lambda},\rho\right)$ verifies $\|g\|_{\ell^{\infty}\left(\mathcal{F}_{k,\lambda}\right)}=\|g\|_{\mathcal{H}_{k,\lambda}^{\ast}}=\sqrt{g\left(g\left(k_{\lambda}\left(\cdot_{1},\cdot_{2}\right)\right)\right)}$. 
			\end{Lema}
			\begin{proof}
				The inclusion $\mathcal{C}_{\operatorname{bpl}}\left(\mathcal{F}_{k,\Lambda},\rho\right)\subseteq\mathcal{H}_{k,\lambda}^{\ast}$ is given by the fact given \ref{itm:Par}, for $f_{1},f_{2}\in\mathcal{F}_{k,\lambda}$
				\begin{equation}
					\rho\left(f_{1},f_{2}\right)\leq\underset{\operatorname{S}\in\{\operatorname{P},\operatorname{Q}\}}{\operatorname{max}}\,\left(\left(\int_{\mathcal{X}}k(x,x)\,\operatorname{d}\!\operatorname{S}(x)\right)^{\rfrac{1}{2}}\right)\,\left\|f_{1}-f_{2}\right\|_{\mathcal{H}_{k,\lambda}}.
				\end{equation}
				
				The expression $\|g\|_{\ell^{\infty}\left(\mathcal{F}_{k,\lambda}\right)}=\|g\|_{\mathcal{H}_{k,\lambda}^{\ast}}$ is a direct consequence of \textcite[Lemma 2.30, p. 88]{Dudley1999}. Now, by Riesz's representation theorem (see \textcite[Th. 3.4]{Conway2019} there exists an element in $\mathcal{H}_{k,\lambda}$, let's call it $\mu_{g}$, such that $\|g\|_{\mathcal{H}_{k,\lambda}^{\ast}}=g\left(\mu_{g}\right)$. Let's prove that $\mu_{g}\left(\cdot_{2}\right)=g\left(k_{\lambda}\left(\cdot_{1},\cdot_{2}\right)\right)$. By \textcite[Th. 3]{Berlinet&Thomas-Agnan2011}, it is enough to check that for every $x\in\mathcal{X}$
				\begin{equation}
					\inp{\mu_{g}}{k_{\lambda}\left(x,\cdot_{2}\right)}{\mathcal{H}_{k,\lambda}}=g\left(k_{\lambda}\left(\cdot_{1},x\right)\right).
				\end{equation}
				Now, note that it is immediate by the reproducing property.
			\end{proof}
			{\color{orange}Comentarios varios sobre que $\mu_{g}$ es el \enquote{mean embedding} de $g$ y que hacer producto escalar contra $\mu_{g}$ es aplicar $g$ pero que no tenemos inter\'{e}s en esas propiedades en este trabajo.}
	\subsection{State of the art: different approaches to parametric families of kernels}
		\begin{itemize}
			\item Median.
			\item Argmax.
			\item Rayleigh's quotient.
		\end{itemize}
	\subsection{Our contribution}
		\begin{itemize}
			\item Asymptotic result for data-driven kernel distance.
			\item Theoretical insight on median heuristic for Gaussian kernel. Revise literature to complement other results on consistency (if they exists or they are formal).
			\item Our proposal: data-driven corrected estimation (with the true asymptotic distribution).
		\end{itemize}
\section{Main results}
	\subsection{Differentiability results}
		{\color{orange} Previous lemma to make life easier later}.
		\begin{Lema}\label{Lema:PartialLipschitz}
			{\color{orange}Resultado sobre la propiedad Lipschitz marginal (extensi\'{o}n del lema de Shapiro).}
		\end{Lema}
		\begin{proof}
			
		\end{proof}
		\subsubsection*{\color{orange} Differentiability under the null. Continuity of $\psi$}
			\begin{Th}\label{Th:Diff0}
				Let us assume that the family of kernels $\left\{k_{\theta}:\theta\in\Lambda\right\}$ satisfies \ref{itm:Dom} and \ref{itm:Par}. The mapping $\psi$ in \eqref{Eqn:ProfCarcaIdea} is Hadamard directionally differentiable at $(\lambda,0)$ tangentially to $\Lambda\times\mathcal{C}_{\operatorname{bpl}}\left(\mathcal{F}_{k,\Lambda},\rho\right)$. In such a case, the (directional) derivative of $\psi$ at $(\lambda,0)$ is given by
				\begin{equation}\label{Eqn:psi_prime0}
					\psi_{(\lambda,0)}^{\prime}(\zeta,g)=\|g\|_{\ell^{\infty}\left(\mathcal{F}_{k,\lambda}\right)}=\|g\|_{\mathcal{H}_{k,\lambda}^{\ast}},
				\end{equation}
				with $g\in\mathcal{C}_{\operatorname{bpl}}\left(\mathcal{F}_{k,\Lambda},\rho\right)$.
			\end{Th}
			\begin{proof}
				By definition of Hadamard directional differentiability and Lemma \ref{Lema:PartialLipschitz}, given $g\in\mathcal{C}_{\operatorname{bpl}}\left(\mathcal{F}_{k,\Lambda},\rho\right)$ and $\zeta\in\Lambda$ and sequences $\left(\zeta_{j}\right)_{j\in\mathbb{N}}\in\Lambda^{\mathbb{N}}$ and $\left(t_{j}\right)_{j\in\mathbb{N}}\in\mathbb{R}^{\mathbb{N}}$ such that $d_{\Lambda}\left(\zeta_{j},\zeta\right)\longrightarrow0$ and $t_{j}\searrow0$ when $j\longrightarrow\infty$ we have to show that
				\begin{equation}
					\lim_{j\longrightarrow\infty}\frac{\psi\left(\lambda+t_{j}\,\zeta_{j},t_{j}\,g\right)-\psi(\lambda,0)-t_{j}\,\psi_{(\lambda,0)}^{\prime}\left(\zeta,g\right)}{t_{j}}=0.
				\end{equation}
				Under the hypothesis of the theorem, note that
				\begin{equation}
					\frac{\psi\left(\lambda+t_{j}\,\zeta_{j},t_{j}\,g\right)-\psi(\lambda,0)-t_{j}\,\psi_{(\lambda,0)}^{\prime}\left(\zeta,g\right)}{t_{j}}=\|g\|_{\ell^{\infty}\left(\mathcal{F}_{k,\lambda+t_{j}\,\zeta_{j}}\right)}-\|g\|_{\ell^{\infty}\left(\mathcal{F}_{k,\lambda}\right)}.
				\end{equation}
				Now observe that $\|g\|_{\ell^{\infty}\left(\mathcal{F}_{k,\lambda+t_{j}\,\zeta_{j}}\right)}=\sqrt{g\left(g\left(k_{\lambda+t_{j}\,\zeta_{j}}\left(\cdot_{1},\cdot_{2}\right)\right)\right)}$. By \ref{itm:Par} and dominated convergence theorem, $k_{\lambda+t_{j}\,\zeta_{j}}\left(\cdot_{1},y\right)$ converges to $k_{\lambda}\left(\cdot_{1},y\right)$ on the metric $\rho$ for every $y\in\mathcal{X}$. Hence, by continuity of $g$, $g\left(k_{\lambda+t_{j}\,\zeta_{j}}\left(\cdot_{1},y\right)\right)$ to $g\left(k_{\lambda}\left(\cdot_{1},y\right)\right)$ pointwise in $y\in\mathcal{X}$.
				
				At this point, it is worth to mention that by \ref{itm:Par}
				\begin{equation}
					\left|g\left(k_{\lambda+t_{j}\,\zeta_{j}}\left(\cdot_{1},y\right)\right)-g\left(k_{\lambda}\left(\cdot_{1},y\right)\right)\right|\leq2\,\|g\|_{\ell^{\infty}\left(\mathcal{F}_{k,\Lambda}\right)}\,\sqrt{G_{1}(y)},
				\end{equation}
				so, by dominated convergence theorem the convergence of $g\left(k_{\lambda+t_{j}\,\zeta_{j}}\left(\cdot_{1},\cdot_{2}\right)\right)$ to $g\left(k_{\lambda}\left(\cdot_{1},\cdot_{2}\right)\right)$ is also given in the metric $\rho$. Thanks to the continuity of $g$, the first part of the proof is ended.
			\end{proof}
		\subsubsection*{\color{orange} Differentiability under the alternative}
			\begin{Th}
				Let us assume that the family of kernels $\left\{k_{\theta}:\theta\in\Lambda\right\}$ satisfies \ref{itm:Ide} and \ref{itm:Par}. If $\operatorname{P},\operatorname{Q}\in\mathcal{MB}_{\operatorname{p}}(\mathcal{X})$ such that $\operatorname{P}\neq\operatorname{Q}$, then the mapping $\psi$ in \eqref{Eqn:ProfCarcaIdea} is Hadamard directionally differentiable at $(\lambda,\operatorname{P}-\operatorname{Q})$ tangentially to $\Lambda\times\mathcal{C}_{\operatorname{bpl}}\left(\mathcal{F}_{k,\Lambda},\rho\right)$,
				% Yo esto lo quitaría por redundante con la definición que estará en la versión final.
				{\color{orange}the subset of $\ell^{\infty}\left(\mathcal{F}_{k,\Lambda}\right)$ constituted by bounded, prelinear and continuous functionals with respect to the distance $\rho$ in \eqref{eq:metric-rho}}. In such a case, the (directional) derivative of $\psi$ at $(\lambda,\operatorname{P}-\operatorname{Q})$ is given by
				\begin{equation}\label{Eqn:psi_prime1}
					\begin{aligned}
						\psi_{(\lambda,\operatorname{P}-\operatorname{Q})}^{\prime}(\zeta,g)&=g\left(h^{+,\lambda}\right)
						\\
						&\quad+\frac{1}{2\,\left\|\mu_{\operatorname{P}-\operatorname{Q}}^{\lambda}\right\|_{\mathcal{H}_{k,\lambda}}}\,\int_{\mathcal{X}}\int_{\mathcal{X}}\partial k_{\lambda}(x,y)(\zeta)\,\operatorname{d}(\operatorname{P}-\operatorname{Q})(y)\,\operatorname{d}(\operatorname{P}-\operatorname{Q})(x),
					\end{aligned}
				\end{equation}
				with $g\in\mathcal{C}_{\operatorname{bpl}}\left(\mathcal{F}_{k,\Lambda},\rho\right)$ and $\zeta\in\Lambda$; where the functions $h^{+,\lambda}$ are defined in \eqref{Eqn:Witness}.
			\end{Th}
			\begin{proof}
				By definition of Hadamard directional differentiability and Lemma \ref{Lema:PartialLipschitz}, given $g\in\mathcal{C}_{\operatorname{bpl}}\left(\mathcal{F}_{k,\Lambda},\rho\right)$ and $\zeta\in\Lambda$ and sequences $\left(\zeta_{j}\right)_{j\in\mathbb{N}}\in\Lambda^{\mathbb{N}}$ and $\left(t_{j}\right)_{j\in\mathbb{N}}\in\mathbb{R}^{\mathbb{N}}$ such that $d_{\Lambda}\left(\zeta_{j},\zeta\right)\longrightarrow0$ and $t_{j}\searrow0$ when $j\longrightarrow\infty$ we have to show that
				\begin{equation}
					\lim_{j\longrightarrow\infty}\frac{\psi\left(\lambda+t_{j}\,\zeta_{j},\operatorname{P}-\operatorname{Q}+t_{j}\,g\right)-\psi(\lambda,\operatorname{P}-\operatorname{Q})-t_{j}\,\psi_{(\lambda,\operatorname{P}-\operatorname{Q})}^{\prime}\left(\zeta,g\right)}{t_{j}}=0.
				\end{equation}
				To begin with, note that
				\begin{equation}
					\frac{\psi\left(\lambda+t_{j}\,\zeta_{j},\operatorname{P}-\operatorname{Q}+t_{j}\,g\right)-\psi(\lambda,\operatorname{P}-\operatorname{Q})-t_{j}\,\psi_{(\lambda,\operatorname{P}-\operatorname{Q})}^{\prime}\left(\zeta,g\right)}{t_{j}}=L_{1}+L_{2}+L_{3},
				\end{equation}
				where
				\begin{equation}
					\begin{aligned}
						L_{1}&=\frac{\underset{f\in\mathcal{F}_{k,\lambda}}{\operatorname{sup}}\,\left(\left(\operatorname{P}-\operatorname{Q}+t_{j}\,g\right)\,(f)\right)-\underset{f\in\mathcal{F}_{k,\lambda}}{\operatorname{sup}}\,\left(\left(\operatorname{P}-\operatorname{Q}\right)\,(f)\right)}{t_{j}},
						\\
						L_{2}&=\frac{1}{L_{4}}\,\int_{\mathcal{X}}\int_{\mathcal{X}}\frac{k_{\lambda+t_{j}\,\zeta_{j}}(x,y)-k_{\lambda}(x,y)}{t_{j}}\,\operatorname{d}(\operatorname{P}-\operatorname{Q})(y)\,\operatorname{d}(\operatorname{P}-\operatorname{Q})(x),
						\\
						L_{3}&=\frac{1}{L_{4}}\,\left(g\left(\mu_{\operatorname{P}-\operatorname{Q}}^{\lambda+t_{j}\,\zeta_{j}}-\mu_{\operatorname{P}-\operatorname{Q}}^{\lambda}\right)+\int_{\mathcal{X}}g\left(k_{\lambda+t_{j}\,\zeta_{j}}(\cdot,y)-k_{\lambda}(\cdot,y)\right)\,\operatorname{d}(\operatorname{P}-\operatorname{Q})(y)\right.
						\\
						&\quad\left.+t_{j}\,\left(\|g\|_{\ell^{\infty}\left(\mathcal{F}_{k,\lambda+t_{j}\,\zeta_{j}}\right)}^{2}-\|g\|_{\ell^{\infty}\left(\mathcal{F}_{k,\lambda}\right)}^{2}\right)\right),
						\\
						L_{4}&=\left(\left\|\mu_{\operatorname{P}-\operatorname{Q}+t_{j}\,g}^{\lambda+t_{j}\,\zeta_{j}}\right\|_{\mathcal{H}_{k,\lambda+t_{j}\,\zeta_{j}}}+\left\|\mu_{\operatorname{P}-\operatorname{Q}+t_{j}\,g}^{\lambda}\right\|_{\mathcal{H}_{k,\lambda}}\right)^{\rfrac{1}{2}}.
					\end{aligned}
				\end{equation}
				The convergence of $L_{1}$ was proved in \textcite[Lemma 4]{Carcamo2024} and the limit is $g\left(h^{+,\lambda}\right)$, where $h^{+,\lambda}$ was defined in \eqref{Eqn:Witness}. Provided the convergence of $L_{4}$ to $2\,\left\|\mu_{\operatorname{P}-\operatorname{Q}}^{\lambda}\right\|_{\mathcal{H}_{k,\lambda}}$, by \ref{itm:Par} and \textcite[Theorem 2.27]{Folland2013}, $L_{2}$ converges to
				\begin{equation}
					\frac{1}{2\,\left\|\mu_{\operatorname{P}-\operatorname{Q}}^{\lambda}\right\|_{\mathcal{H}_{k,\lambda}}}\,\int_{\mathcal{X}}\int_{\mathcal{X}}\partial k_{\lambda}(x,y)(\zeta)\,\operatorname{d}(\operatorname{P}-\operatorname{Q})(y)\,\operatorname{d}(\operatorname{P}-\operatorname{Q})(x).
				\end{equation}
				
				Now we continue with $L_{4}$. By definition
				\begin{equation}\label{Eqn:L4}
					\begin{aligned}
						\left\|\mu_{\operatorname{P}-\operatorname{Q}+t_{j}\,g}^{\lambda+t_{j}\,\zeta_{j}}\right\|_{\mathcal{H}_{k,\lambda+t_{j}\,\zeta_{j}}}&=\left(\int_{\mathcal{X}}\int_{\mathcal{X}}k_{\lambda+t_{j}\,\zeta_{j}}(x,y)\,\operatorname{d}(\operatorname{P}-\operatorname{Q})(y)\,\operatorname{d}(\operatorname{P}-\operatorname{Q})(x)\right.
						\\
						&\quad+t_{j}\,g\left(\mu_{\operatorname{P}-\operatorname{Q}}^{\lambda+t_{j}\,\zeta_{j}}\right)
						\\
						&\quad+t_{j}\,\int_{\mathcal{X}}g\left(k_{\lambda+t_{j}\,\zeta_{j}}(\cdot,y)\right)\,\operatorname{d}(\operatorname{P}-\operatorname{Q})(y)
						\\
						&\quad\left.+t_{j}^{2}\,\|g\|_{\ell^{\infty}\left(\mathcal{F}_{k,\lambda+t_{j}\,\zeta_{j}}\right)}\right)^{\rfrac{1}{2}}.
					\end{aligned}
				\end{equation}
				From top to bottom in \eqref{Eqn:L4}:
				\begin{enumerate}
					\item $\displaystyle\int_{\mathcal{X}}\int_{\mathcal{X}}k_{\lambda+t_{j}\,\zeta_{j}}(x,y)\,\operatorname{d}(\operatorname{P}-\operatorname{Q})(y)\,\operatorname{d}(\operatorname{P}-\operatorname{Q})(x)$ converges to\break$\displaystyle\int_{\mathcal{X}}\int_{\mathcal{X}}k_{\lambda}(x,y)\,\operatorname{d}(\operatorname{P}-\operatorname{Q})(y)\,\operatorname{d}(\operatorname{P}-\operatorname{Q})(x)=\left\|\mu_{\operatorname{P}-\operatorname{Q}}^{\lambda}\right\|_{\mathcal{H}_{k,\lambda}}^{2}$ by \ref{itm:Par} and Cauchy-Schwarz's inequality and dominated convergence theorem.
					\item By $g\in\mathcal{C}_{\operatorname{bpl}}\left(\mathcal{F}_{k,\Lambda},\rho\right)$
					\begin{equation}
						\left|g\left(\mu_{\operatorname{P}-\operatorname{Q}}^{\lambda+t_{j}\,\zeta_{j}}\right)\right|\leq\|g\|_{\ell^{\infty}\left(\mathcal{F}_{k,\lambda+t_{j}\,\zeta_{j}}\right)}\,\left\|\mu_{\operatorname{P}-\operatorname{Q}}^{\lambda}\right\|_{\mathcal{H}_{k,\lambda}},
					\end{equation}
					so by \ref{itm:Par},
					\begin{equation}
						\left|g\left(\mu_{\operatorname{P}-\operatorname{Q}}^{\lambda+t_{j}\,\zeta_{j}}\right)\right|\leq\|g\|_{\ell^{\infty}\left(\mathcal{F}_{k,\Lambda}\right)}\,\int_{\mathcal{X}}\sqrt{G_{1}(x)}\,\operatorname{d}(\operatorname{P}+\operatorname{Q})(y),
					\end{equation}
					and $t_{j}\,g\left(\mu_{\operatorname{P}-\operatorname{Q}}^{\lambda+t_{j}\,\zeta_{j}}\right)$ goes to $0$.
					\item Analogously, $\left|g\left(k_{\lambda+t_{j}\,\zeta_{j}}(\cdot,y)\right)\right|\leq\|g\|_{\ell^{\infty}\left(\mathcal{F}_{k,\Lambda}\right)}\,\sqrt{G_{1}(y)}$, so we can conclude that $\displaystyle t_{j}\,\int_{\mathcal{X}}g\left(k_{\lambda+t_{j}\,\zeta_{j}}(\cdot,y)\right)\,\operatorname{d}(\operatorname{P}-\operatorname{Q})(y)$ is also tending to $0$.
					\item Finally, since $\|g\|_{\ell^{\infty}\left(\mathcal{F}_{k,\lambda+t_{j}\,\zeta_{j}}\right)}\leq\|g\|_{\ell^{\infty}\left(\mathcal{F}_{k,\Lambda}\right)}$, then $t_{j}^{2}\,\|g\|_{\ell^{\infty}\left(\mathcal{F}_{k,\lambda+t_{j}\,\zeta_{j}}\right)}$ converges to $0$.
				\end{enumerate}
				In conclusion, $L_{4}\longrightarrow2\,\left\|\mu_{\operatorname{P}-\operatorname{Q}}^{\lambda}\right\|_{\mathcal{H}_{k,\lambda}}$ when $j\longrightarrow\infty$. To end up this proof, we see the convergence of $L_{3}$. Firstly, we tackle $\displaystyle\int_{\mathcal{X}}g\left(k_{\lambda+t_{j}\,\zeta_{j}}(\cdot,y)-k_{\lambda}(\cdot,y)\right)\,\operatorname{d}(\operatorname{P}-\operatorname{Q})(y)$. Pointwise convergence of $k_{\lambda+t_{j}\,\zeta_{j}}(\cdot,y)$ to $k_{\lambda}(\cdot,y)$ is given by \ref{itm:Par} and \textcite[Theorem 2.27]{Folland2013}, convergence in metric $\rho$ is also given. By continuity of $g$, $g\left(k_{\lambda+t_{j}\,\zeta_{j}}(\cdot,y)-k_{\lambda}(\cdot,y)\right)$ is converging to $0$ pointwise when $j\longrightarrow\infty$. Since this expression is also bounded by the third item of the previous enumeration, by dominated convergence theorem we have the desired limit.
				
				Secondly, convergence $g\left(\mu_{\operatorname{P}-\operatorname{Q}}^{\lambda+t_{j}\,\zeta_{j}}-\mu_{\operatorname{P}-\operatorname{Q}}^{\lambda}\right)$ to $0$ is to be proved. By \ref{itm:Par}, and definition of meam embedding, $\mu_{\operatorname{P}-\operatorname{Q}}^{\lambda+t_{j}\,\zeta_{j}}$ converges to $\mu_{\operatorname{P}-\operatorname{Q}}^{\lambda}$ pointwise. Now, recall that
				\begin{equation}
					\left|\mu_{\operatorname{P}-\operatorname{Q}}^{\lambda+t_{j}\,\zeta_{j}}(x)-\mu_{\operatorname{P}-\operatorname{Q}}^{\lambda}(x)\right|\leq2\,\sqrt{G_{1}(x)}\,\int_{\mathcal{X}}\sqrt{G_{1}(y)}\,\operatorname{d}(\operatorname{P}+\operatorname{Q})(y),
				\end{equation}
				by \ref{itm:Par}. So, by virtue of dominated convergence theorem, the convergence is also given in the metric $\rho$. By continuity of $g$, this term is also done.
				
				Finally, for the last term it is enough to observe that $\left|\|g\|_{\ell^{\infty}\left(\mathcal{F}_{k,\lambda+t_{j}\,\zeta_{j}}\right)}^{2}-\|g\|_{\ell^{\infty}\left(\mathcal{F}_{k,\lambda}\right)}^{2}\right|\leq2\,\|g\|_{\ell^{\infty}\left(\mathcal{F}_{k,\Lambda}\right)}^{2}$. Hence, $L_{3}$ is converging to $0$ and the proof is ended.
			\end{proof}
	\subsection{Statistic results}
			\subsubsection*{Our asymptotic results}
				{\color{orange}Delta method y palante.}
	\subsection{Empirical results}	
\section{Notas}
	\begin{enumerate}
		\item What happens when the estimated parameter goes to $0$ or $\infty$ in the Gaussian kernel? The limit of the estimated parameter should belong to the parameter space (see Theorem \ref{Th:Diff0}).
		\item What is the new process? Obviously the empirical process is involved in the second argument. But for the first we should have to add assumptions on the parameter estimation (M-estimators, etc).
		\item Empirical results, code (\verb!C++!) and so: having the asymptotic distribution under the alternative, we can detect or explore examples where Gretton's heuristics is not working (interaction between the two terms of the limit, see below).
	\end{enumerate}